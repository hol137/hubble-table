\documentclass{article}
\usepackage[utf8]{inputenc}

%\title{hubble}
%\author{hol137 }
%\date{January 2020}

\begin{document}

%\maketitle

%\section{Introduction}
\begin{table*}
  \hskip-4.5cm\begin{tabular}{llll}
    \toprule
    \multicolumn{4}{c}{Tableau 1.Quinze formules au centi\`{e}me pr\`{e}s pour le rayon de Hubble-Lema\^{i}tre dont une au millionième}\\
    \cmidrule
   \#     & Formule     & Valeur(Gal) & Remarques \\
    \midrule
    1 & $2\hbar^2/Gm_em_pm_n$ & 13.80 & Formule obtenue en 3mn (Sept. 1997) par analyse dimensionelle sans c \\
    2 & $2\hbar^2/Gm_em_p^2$ & 13.82 & Rayon th\'{e}orique d'\'{e}toile monoatomique \\
    3 & $\lambdabar_e g(6)$ & 13.82 & Fonction Topologique $g(k)=exp(2^{k+1/2})/k$ pour k=6 (d=26 valeur critique)\\
    4 & $(2\lambdabar_{e}/3)(\lambdabar_{CMB}/\lambdabar_{H2})^3$ & 13.90 & Conservation holographique Spéciale dans la Molécule Gravitationnelle d'Hydrogène \\
    5 & $\lambdabar_e s_4^5$ & 13.80 & $s_4=2\pi^2a^3$: prolongement 5D de la Conservation Holographique Spéciale \\
    6 & $\lambdabar_p (WZ)^{4}$ & 13.80 & prècise la relation $a_G \approx W^8$ \\
    7 & $\lambdabar_e 2^{128}$ & 13.90 & $R/2 \approx 2^{127}-1$: nombre de Lucas dernier terme de la hi\'{e}rarchie combinatoire \\ 
    8 & $\lambdabar_e \pi^{155/2}$ & 13.80 & $\pi$ base de calcul, comme dans les s\'{e}ries de Riemann: $2^{1/155} \approx \pi^{1/256} \approx (2\pi)^{1/(3\times 137)}$ \\
    
    9 & $\lambda_e O_M^{7/10}$ & 13.94 & provient de $O_M^{7/10} \approx 496$, dimension du groupe de jauge supercorde SO32 \\
    10 & $(\lambdabar_{Ryd} n^{4})^2/\lambdabar_p$ & 13.81 &  provient de $ct_K/\lambdabar_e \approx aFWZn$ \\ 
    11 & $(\lambda_{CMB}/(j+1))^2/l_P$ & 13.80 & provient de la relation centrale cosmo-biologique: $\sqrt(Rl_P) \approx \lambda_{mam}$ \\
    12 & $(2/u)N_{Edd}Gm_H/c^2$ & 13.79 & confirme le nombre d'Eddington et l'existence de la masse noire  \\
    
    13 & $\lambdabar_{e} (2R/R_N)^{210}$ & 13.85 & confirme le principe holique et l'hologramme de rayon  $R_N$ du Grandcosmos   \\
    
    14 & $R_N (O_MO_B/n_{ph})^2\lambdabar_{e} (2R/R_N)^{210}$ & 13.85 & confirme le rôle central des groupes sporadiques   \\
    
    
    15 & $2(ct_K/F)^2/\lambdabar_{e}$ & 13.81198(3) & Après élimination de $c$ entre les couplages gravitationel et interm\'{e}diaire \\
    \bottomrule
  \end{tabular}
  \label{tab:table}
\end{table*}
\begin{table*}
  \hskip-2.0cm\begin{tabular}{llll}
    \toprule
    \multicolumn{4}{c}{Tableau 2. Cinq formules au milliardi\`{e}me pr\`{e}s pour le rayon de Hubble-Lema\^{i}tre}                   \\
 %   \cmidrule(r){1-4}
   \cmidrule 
   \#     & Formule     & Valeur(Gal) & Remarques \\
    \midrule
    1 & $\lambdabar_{e} 2^{128}(1-(137^2+\pi^2+e^2)/pH)$ & 13.8119768  & Montre la sym\'{e}trie de $\pi$, e and 137 \\
    2 & $ \lambdabar_{e} 2^{137} (\gamma n^3)^2/137^3 \Gamma^{11}$ & 13.8119768  & raccorde en 9D avec la constante d'Atiyah \Gamma \\
    3 & $\lambdabar_{e} 2^{128}/d_e^2(m_H/m_p)^6$ & 13.8119768  & empirique \\
    4 & $\lambdabar_e g(6)/(1+\sqrt(137^2+\sqrt(136))/jn)$ & 13.8119768 & Confirmation de 137=136+1 \\
    5 & $2\lambdabar_{e} (pn/H^{2})(g(5)/\ln(2-1/ja^2))^2$ & 13.8119767  & implique la pertinence de $\ln(2) \approx 2\sqrt(3/5)$  \\    
    %5 & $F_{AS}^5/a^3 \approx \eta \lambdabar_{e}/l_P$ & 13.8119767  & Relation Centrale  \\
    \bottomrule
  \end{tabular}
  \label{tab:table}
\end{table*}
\end{document}

