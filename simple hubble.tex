\documentclass[a4paper,9pt]{article}
\usepackage{conference}
\usepackage{latexsym}
\usepackage[utf8]{inputenc}
\usepackage[english]{babel}
\usepackage{amssymb,amsfonts,amsmath}
\usepackage{graphicx}
\usepackage{pstricks}
\usepackage{cite}
\usepackage{hyperref}
\usepackage[varg]{txfonts}

\usepackage{booktabs}       % professional-quality tables

%\usepackage[letterpaper, landscape, lmargin=0.25in, rmargin=1.25in]{geometry}
\usepackage{tabularx}
\usepackage{enumerate}
%\usepackage{enumitem}
\usepackage{multicol}
\pagenumbering{gobble}
\title{Hubble Table}
%\author{author}
\date{01/20/2020}

\title{Hubble Table \emph{conference} preprint}
%{F.M. Sanchez,\ M. Grosmann,\ B. Kress,\ N. Flawisky,\ L. Gueroult, \textit{Cosmic Holography}}

  %% \AND
  %% Coauthor \\
  %% Affiliation \\
  %% Address \\
  %% \texttt{email} \\
  %% \And
  %% Coauthor \\
  %% Affiliation \\
  %% Address \\
  %% \texttt{email} \\
  %% \And
  %% Coauthor \\
  %% Affiliation \\
  %% Address \\
  %% \texttt{email} \\
}


\begin{document}
\maketitle

\begin{abstract}
\end{abstract}

\begin{table*}
  \hskip-2.0cm\begin{tabular}{llll}
    \toprule
    \multicolumn{4}{c}{14 formules au centi\`{e}me pr\`{e}s pour le rayon de Hubble-Lema\^{i}tre}                   \\
    \cmidrule(r){1-4}
   \#     & Formule     & Valeur(Gal) & Remarques \\
    \midrule
    1 & $2\hbar^2/Gm_em_pm_n$ & 13.80 & Calcul obtenu en 3mn (1997) par analyse dimensionelle sans c \\
    2 & $2\hbar^2/Gm_em_p^2$ & 13.82 & Rayon th\'{e}orique d'\'{e}toile monoatomique \\
    3 & $\lambdabar_e g(6)$ & 13.82 & Fonction topologique $g(k)=exp(2^{k+1/2})/k$ pour k=6 (d=26 valeur critique)\\
    4 & $\lambdabar_e s_4^5$ & 13.80 & $s_4=2\pi^2a^3$ aire de la sph\`{e}re 4D de rayon $a \approx 137.036$ \\  
    5 & $\lambdabar_e 2^{128}$ & 13.90 & $R/2 \approx 2^{127}$ Nombre de Lucas dernier terme de la hi\'{e}rarchie combinatoire \\ 
    6 & $\lambdabar_e \pi^{155/2}$ & 13.80 & $\pi$ base de calcul comme dans les s\'{e}ries de Riemann: $2^{1/155} \approx \pi^{1/256} \approx (2\pi)^{1/(3\times 137)}$ \\
    7 & $\lambdabar_p (WZ)^{4}$ & 13.80 & relation $a_G \approx W^8$ [3] \\
    8 & $\lambda_e O_M^{7/10}$ & 13.94 & $O_M^{7/10} \approx 496$, dimension du groupe de jauge supercorde SO32 \\
    9 & $(\lambdabar_{Ryd} n^{4})^2/\lambdabar_p$ & 13.81 & vient de $ct_K/\lambdabar_e \approx aFWZn$ \\ 
    10 & $(2\lambdabar_{e}/3)(\lambdabar_{CMB}/\lambdabar_{H2})^3$ & 13.90 & Conservation holographique \\
    11 & $(\lambda_{CMB}/(j+1))^2/l_P$ & 13.80 & vient de la relation centrale cosmo-biologique [3]: $\sqrt(Rl_P) \approx \lambda_{mam}$ \\
    12 & $(20/3)N_{Edd}Gm_H/c^2$ & 13.79 & Confirme Eddington et l'existence de la masse noire [3] \\
    13 & $\lambdabar_{e} (2R/R_N)^{210}$ & 13.85 & Confirme le principe holique et l'hologramme du Grandcosmos de rayon $R_N$  \\
    14 & $2(ct_K/F)^2/\lambdabar_{e}$ & 13.81198(3) & Elimination de $c$ entre les couplages gravitationel et interm\'{e}diaire [4] \\
    \bottomrule
  \end{tabular}
  \label{tab:table}
\end{table*}

\begin{table*}
  \hskip-2.0cm\begin{tabular}{llll}
    \toprule
    \multicolumn{4}{c}{4 formules au milliardi\`{e}me pr\`{e}s pour le rayon de Hubble-Lema\^{i}tre}                   \\
    \cmidrule(r){1-4}
   \#     & Formule     & Valeur(Gal) & Remarques \\
    \midrule
    1 & $\lambdabar_{e} 2^{128}(1-(137^2+\pi^2+e^2)/pH)$ & 13.8119768  & Montre la sym\'{e}trie de $\pi$, e and 137 \\
    2 & $\lambdabar_{e} 2^{128}/d_e^2(m_H/m_p)^6$ & 13.8119768  & Raccorde la constante d'Atiyah \\
    3 & $\lambdabar_e g(6)/(1+\sqrt(137^2+\sqrt(136))/jn)$ & 13.8119768 & Confirmation de 137=136+1 \\
    4 & $2\lambdabar_{e} (pn/H^{2})(g(5)/\ln(2-1/ja^2))^2$ & 13.8119767  & Implique la pertinence de $\ln(2) \approx 2\sqrt(3/5)$  \\    
    %5 & $F_{AS}^5/a^3 \approx \eta \lambdabar_{e}/l_P$ & 13.8119767  & Relation Centrale  \\
    \bottomrule
  \end{tabular}
  \label{tab:table}
\end{table*}

\end{document}
