\documentclass{article}
\usepackage{conference}

\usepackage[utf8]{inputenc} % allow utf-8 input
\usepackage[T1]{fontenc}    % use 8-bit T1 fonts
\usepackage{hyperref}       % hyperlinks
\usepackage{url}            % simple URL typesetting
\usepackage{booktabs}       % professional-quality tables
\usepackage{amsfonts}       % blackboard math symbols
\usepackage{nicefrac}       % compact symbols for 1/2, etc.
\usepackage{microtype}      % microtypography
\usepackage[final]{graphicx}% graphics
\usepackage{lipsum}

\title{Hubble Table \emph{conference} preprint}
%{F.M. Sanchez,\ M. Grosmann,\ B. Kress,\ N. Flawisky,\ L. Gueroult, \textit{Cosmic Holography}}
\author{
  F.M. Sanchez~hol137\thanks{Use footnote for providing further
    information about author (webpage, alternative
    address)---\emph{not} for acknowledging funding agencies.} \\
  Department of Physics\\
  Paris 11 University\\
  Paris, FRANCE \\
  \texttt{hol137@yahoo.fr} \\
  %% examples of more authors
   \And
 M.H. Grosmann \\
  Department of Photonics\\
  University of Strasbourg\\
  Strasbourg, FRANCE \\
  \texttt{michel.grosmann@me.com} \\
  %% \AND
  %% Coauthor \\
  %% Affiliation \\
  %% Address \\
  %% \texttt{email} \\
  %% \And
  %% Coauthor \\
  %% Affiliation \\
  %% Address \\
  %% \texttt{email} \\
  %% \And
  %% Coauthor \\
  %% Affiliation \\
  %% Address \\
  %% \texttt{email} \\
}


\begin{document}
\maketitle

\begin{abstract}
%\lipsum[1]
Hubble Table
\end{abstract}


% keywords can be removed
\keywords{Quantum \and Holography \and Cosmos}


\section{Introduction}

Le modèle dit du Big-Bang est une description de l'origine et de l'évolution de l’Univers datant du début du 20éme siècle.

De façon générale, le terme « Big Bang » est associé à toutes les théories qui décrivent notre Univers comme issu d'une dilatation rapide. Par extension, il est également associé à cette époque dense et chaude qu’aurait connue l’Univers il y a 13,8 milliards d’années, (sans que cela préjuge forcément de l’existence d’un « instant initial » ou d’un commencement à son histoire).

\section{Le modèle du ”Big-Bang”}
\label{sec:headings}

Le terme a été initialement proposé en 1927 par l'astrophysicien  et chanoine catholique belge Georges Lemaître, qui décrivait dans les grandes lignes l’expansion de l'Univers, plus précisément décrite par l'astronome américain Edwin Hubble en 1929. Ce modèle fut désigné pour la première fois sous le terme ironique de « Big Bang » lors d’une émission de la BBC, The Nature of Things en 1949 (dont le texte fut publié en 1950), par le physicien britannique Fred Hoyle. Lui-même préférait les modèles d'état stationnaire.

Voir Section \ref{sec:headings}.

\subsection{Concept General}
%\lipsum[5]
Le concept général du Big Bang, à savoir que l’Univers est en expansion et a été plus dense et plus chaud par le passé, doit sans doute être attribué au Russe Alexandre Friedmann, qui l'avait proposé en 1922, cinq ans avant Lemaître. Son assise fut cependant considérée comme mieux établie en 1965 avec la découverte du fond diffus cosmologique. Georges Lemaître le qualifia d’« éclat disparu de la formation des mondes, attestant de façon définitive la réalité de l’époque dense et chaude de l’Univers primordial ». Albert Einstein, en mettant au point la relativité générale, aurait pu déduire l'expansion de l'Univers, mais a préféré modifier ses équations en y ajoutant sa constante cosmologique, car il était persuadé que l'Univers devait être statique.

\subsubsection{Partisans et adversaires: la controverse}
%\lipsum[6]

Partisans et adversaires du modèle du Big Bang.

\section{Le ”modèle Cosmhologique"}
\label{sec:headings}

Dans ce modèle il n'y a pas de « début » ni de « fin » de l'Univers.


\section{Conclusions}
\label{sec:headings}

Les récentes découvertes expérimentales en macro- et micro- Physique permettent d'imaginer un modèle d'Univers qui résoudrait élégamment les contradictions actuelles entre ces deux sous-disciplines. De nouvelles expériences (en préparation) devraient permettre de confirmer la pertinence de ce modèle face aux très nombreux autres actuellement en compétition. Nous essayons actuellement de le représenter sous forme d'un hologramme. Dans la conception de celui-ci nous avons réalisé des stéréoscopies dont l'une est présentée sur la Figure.

On y voit la représentation (étonnement sous forme d'une droite!) de divers paramètres tant de la MACRO- que de la micro- Physique.


\subsection{Figures}

Voir Figure \ref{fig:figure_label}. Pour plus d'explications. \footnote{http://vixra.org/abs/1904.0218}
 

\begin{table}[]
\begin{tabular}{|l|l|l|l|}
\hline
\multicolumn{2}{|c|}{Hubble Table}          & \multicolumn{1}{r|}{}                                                                     &                                                                                         \\ \hline
Description & \multicolumn{1}{c|}{Equation} & \multicolumn{1}{c|}{\begin{tabular}[c]{@{}c@{}}Value in Gly\\ approximation\end{tabular}} & \multicolumn{1}{c|}{\begin{tabular}[c]{@{}c@{}}Value of G\\ approximation\end{tabular}} \\ \hline
Example 1   & $\lambdabar_{e} \cdot g(6)$                    & \multicolumn{1}{r|}{13.82}                                                                & 6.67.10\textasciicircum{}-11                                                            \\ \hline
Example 2   & $\lambdabar_{e} \cdot s_{4}^{5}$                    & \multicolumn{1}{r|}{13.80}                                                                 & 6.67.10\textasciicircum{}-11                                                            \\ \hline
Example 3   & $\lambdabar_{e} \cdot 2^{128}$                    & \multicolumn{1}{r|}{13.90}                                                                   & 6.67.10\textasciicircum{}-11                                                            \\ \hline
Example 4   & Equation 4                    & \multicolumn{1}{r|}{13.80}                                                                & 6.67.10\textasciicircum{}-11                                                            \\ \hline
Example 5   & Equation 5                    & \multicolumn{1}{r|}{13.80}                                                                & 6.67.10\textasciicircum{}-11                                                            \\ \hline
Example 6   & Equation 6                    & 13.81                                                                                      & 6.67.10\textasciicircum{}-11                                                            \\ \hline
Example 7   & Equation                      & 13.90                                                                                      & 6.67.10\textasciicircum{}-11                                                            \\ \hline
Example 8   & Equation                      & 13.80                                                                                      & 6.67.10\textasciicircum{}-11                                                            \\ \hline
Example 9   & Equation                      & 13.79                                                                                     & 6.67.10\textasciicircum{}-11                                                            \\ \hline
Example 10  & Equation                      & 13.85                                                                                     & 6.67.10\textasciicircum{}-11                                                            \\ \hline
Example 11  & Equation                      & 13.81198                                                                                     & 6.67.10\textasciicircum{}-11                                                            \\ \hline
Example 12  & Equation 12                   & 13.8119768                                                                                   & 6.67.10\textasciicircum{}-11                                                            \\ \hline
\end{tabular}
\end{table}

\subsection{Lists}

\begin{itemize}
\item Axe Topologique
\end{itemize}


\bibliographystyle{unsrt}  
%\bibliography{references}  %%% Remove comment to use the external .bib file (using bibtex).
%%% and comment out the ``thebibliography'' section.


%%% Comment out this section when you \bibliography{references} is enabled.
\begin{thebibliography}{1}

\bibitem{Sanchez3} Sanchez F.M. ``Towards the grand unified Holic Theory''. Current
Issues in Cosmology. Ed. J.-C. Pecker and J. Narlikar. Cambridge Univ. Press,
2006; p. 257--260.

\bibitem{Sanchez4} Sanchez F; M. ``Holic Principle: The coherence of the Universe`` (Sept 1995), Entelechies, 16th ANPA, 324--344.
\bibitem{Grosmann} Grosmann, M. and Meyrueis P. ``Optics and Photonics Applied to Communication and Processing''. SPIE.  Jan 1979.
\newblock Optics and Photonics Applied to Communication and Processing.
\newblock In {\em SPIE (SPIE), 1979 
  International Conference on}, pages . SPIE, 1979.
\bibitem{Grosmann2} Grosmann, M and Rebordão, José and Meyrueis, Patrick, 1985,02,p761--765,Propagation Of Waves In Optical Systems: Reformulation Of Huyghens Principle For Aspheric Systems,
volume 491, Proceedings of SPIE - The International Society for Optical Engineering, doi:10.1117/12.968010
\bibitem{Kress} Digital Diffractive Optics: An Introduction to Planar Diffractive Optics and Related Technology, by B. Kress, P. Meyrueis, pp. 396. ISBN 0-471-98447-7. Wiley-VCH , October 2000.
\newblock An Introduction to Planar Diffractive Optics and Related Technology.
\bibitem{Sanchez5} F.M. Sanchez, V. Kotov, M. Grosmann, D. Weigel, R. Veysseyre, C. Bizouard, N. Flawisky, D. Gayral, L. Gueroult, Back to Cosmos
\newblock {\em arXiv preprint viXra:1904.0218}, 2019.
\end{thebibliography}
\end{document}
